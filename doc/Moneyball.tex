\documentclass[10pt,letterpaper]{article}
\usepackage[latin1]{inputenc}
\usepackage{amsmath}
\usepackage{amsfonts}
\usepackage{amssymb}
\usepackage{tikz}
\usepackage{graphicx}
\usepackage{caption}
\usepackage{subcaption}
\usepackage[top=1in,bottom=1in,right=1in,left=1in]{geometry}
\begin{document}
\begin{center}
\begin{tikzpicture}[scale=.032]

\coordinate (HomePlate) at (0,0);
\coordinate (LeftFieldFoul) at (-251,251);
\coordinate (RightFieldFoul) at (251,251);
\coordinate (CenterField) at (0,400);
\coordinate (CenterFieldLeft) at (-52.6610, 400);
\coordinate (CenterFieldRight) at (52.6610, 400);
\coordinate (Pitcher) at (0,60.5);
\coordinate (ThirdBase) at (-63.6, 63.6);
\coordinate (FirstBase) at (63.6, 63.6);
\coordinate (SecondBase) at (0,127.3);


\draw (HomePlate) -- (LeftFieldFoul);
\draw (HomePlate) -- (RightFieldFoul);
\draw (LeftFieldFoul) -- (CenterFieldLeft);
\draw (RightFieldFoul) -- (CenterFieldRight);
\draw (CenterFieldLeft) -- (CenterFieldRight);

\draw (ThirdBase) -- (SecondBase);
\draw (FirstBase) -- (SecondBase);

\draw (Pitcher) circle (9) node{1};
\draw (Pitcher) node[left=7pt]{15};
\draw (Pitcher) node[right=7pt]{13};
\draw (-8.4866,8.4866) arc (135:405:12);
\draw (0,38) node {1s};
\draw (0,15) node {2};
\draw (0,-20) node {2f};
\draw (89.7167,89.7167) arc (20:160:95);
\draw (36.416,36.416) arc (45:135:51.5);
\draw (26.5965,64.2096) arc (67.5:112.5:69.5);
\draw (17.68,17.68) arc (45:135:25);

\draw (0,69.5) -- (0,200);
\draw (9.5671,23.0970) -- (138.862,335.242);
\draw (-9.5671, 23.0970) -- (-138.862,335.242);
\draw (-9.0716, 68.9054) -- (CenterFieldLeft);
\draw (9.0716, 68.9054) -- (CenterFieldRight);
\draw (0,0) -- (216.5460, 276.8831);
\draw (0,0) -- (-216.5460, 276.8831);
\end{tikzpicture}
\end{center}
\begin{table}[!h]
\subfloat[Positions]{
\begin{tabular}{rl}
1 & Catcher\\ 
2 & Pitcher\\ 
3 & First Base\\ 
4 & Second Base\\ 
5 & Third Base\\ 
6 & Shortstop\\ 
7 & Left Field\\ 
8 & Center Field\\ 
9 & Right Field\\
\end{tabular}}
\subfloat[Play Type]{
\begin{tabular}{rl}
(none) & Out \\
S & Single \\
D & Double \\
T & Triple \\
HR & Home Run \\
HP & Hit by pitch \\
W & Walk \\
IW & Int. Walk \\
WP & Wild Pitch \\
\end{tabular}}
\subfloat[Description]{
\begin{tabular}{rl}
G & Ground Ball \\
L & Line Drive \\
P & Pop Fly \\
F & Fly Ball \\
B & Prefix indicating bunt \\
\end{tabular}}
\subfloat[Pitches]{
\begin{tabular}{rl}
C & Called Strike \\
S & Swinging Strike \\
B & Ball \\
F & Foul Ball \\
L & Foul Bunt \\
M & Missed Bunt \\
P & Pitchout \\
Q & Pitchout (strike) \\
R & Pitchout (foul) \\
I & Int. Ball \\
H & Hit by pitch \\
K & Strike (unknown) \\
U & Unknown \\
n\footnotemark & Pickoff \\
+n\footnotemark & Pickoff by catcher \\
\end{tabular}}
\end{table}  
\footnotetext[1]{n is 1, 2, or 3, corresponding to base thrown to.  This should not be confused with player number.}
\footnotetext[2]{See above}
Sample Records
\begin{align}
start&,\overbrace{grudm001}^\text{Player ID},\overbrace{"Mark Grudzielanek"}^\text{Player Name},\overbrace{0}^\text{Team},\overbrace{1}^\text{Batting Order},\overbrace{6}^\text{Position} \nonumber \\
play&,\overbrace{1,0}^\text{Inning},\overbrace{aaroh101}^\text{Player ID},\overbrace{12}^\text{Count},\overbrace{BF1CX}^\text{Pitches},\overbrace{\underbrace{S}_\text{Play Type}\underbrace{7}_\text{Location}/\underbrace{L7S}_\text{Description}.\underbrace{3-H;1-2}_\text{Runner movement}}^\text{Play} \nonumber
\end{align}
\begin{description}
\item[Hadoop] A platform for storage and processing of large data sets. Hadoop consists of a distributed filesystem, Java interfaces, and services for managing distribution of tasks. Core code is maintained by the Apache Software Foundation.
\item[MapR distribution for Hadooop] MapR provides an enterprise-ready distribution for Hadoop including the core open source components as well as a high-performance distributed filesystem, high-availability services, and easy-to-use management interface.
\item[MapReduce] A programming model for parallel processing, introduced in Google's 2004 paper\footnote{http://research.google.com/archive/mapreduce.html}. Hadoop provides a framework for running MapReduce jobs written in Java.
\item[Pig] A high level data flow language for processing data. Pig programs describe steps to be executed on sets of tuples, and are executed by running one or more MapReduce jobs.
\item[Hive] A framework for executing SQL-like queries on large data sets. The queries are executed as MapReduce jobs.
\end{description}
\end{document}
